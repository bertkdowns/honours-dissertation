\documentclass[12pt]{article}
\usepackage[a4paper, portrait, margin=2cm]{geometry}
\usepackage{graphicx} % Required for inserting images
\usepackage[page,toc,titletoc,title]{appendix}
\usepackage{mathptmx}
\usepackage{pdflscape}
\usepackage{pdfpages}
\usepackage{enumitem}
\usepackage{hyperref}
\usepackage{cleveref}
\usepackage{wrapfig}
\usepackage{gensymb}
\usepackage{amssymb}
\usepackage{makecell}
\usepackage{pifont}% http://ctan.org/pkg/pifont
\newcommand{\cmark}{\ding{51}}%
\newcommand{\xmark}{\ding{55}}%

\title{Evaluating The Effectiveness of IDAES and the Ahuora Simulation Platform for Live Data Processing}

\author{Bert Downs\\
\textit{Ahuora Research Group}\\
\textit{University of Waikato}\\
\textit{Hamilton, New Zealand}\\
\texttt{bd65@students.waikato.ac.nz}}

\date{October 2024}

\begin{document}

\maketitle

\section*{Abstract}
% A concise and factual abstract is required. The abstract should state briefly the purpose of the research, the principal results and major conclusions. An abstract is often presented separately from the article, so it must be able to stand alone. For this reason, References should be avoided, but if essential, then cite the author(s) and year(s). Also, non-standard or uncommon abbreviations should be avoided, but if essential they must be defined at their first mention in the abstract itself.

% Keywords. Immediately after the abstract, provide a maximum of 6 keywords avoiding general and plural terms and multiple concepts (avoid, for example, "and", "of"). Be sparing with abbreviations: only abbreviations firmly established in the field may be eligible. These keywords will be used for indexing purposes.


\section{Introduction}
% State the objectives of the work and provide an adequate background, avoiding a detailed literature survey or a summary of the results.



This research evaluates the extent to which the IDAES-PSE modelling framework can be used to implement emerging chemical modelling techniques for live data processing, to prove it's suitability as a platform on which to develop such tools.

\section{Method}
%Experimental procedure
%Provide sufficient details to allow the work to be reproduced by an independent researcher. Methods that are already published should be summarized, and indicated by a reference. If quoting directly from a previously published method, use quotation marks and also cite the source. Any modifications to existing methods should also be described.
%This section can contain Material and methods and Theory/calculation. A Theory section should extend, not repeat, the background to the article already dealt with in the Introduction and lay the foundation for further work. In contrast, a Calculation section represents a practical development from a theoretical basis.

IDAES has many components, so an iterative validation process is used. The evaluation is broken down into parts, each focusing on a different piece of functionality. 

The core functionality of the IDAES modelling framework is steady-state modelling, where a chemical system is in a stable equilibrium and variables do not change over time. In a live data processing context, steady-state simulations can be used to model the entire state of the factory, based on a sample of avaliable sensor data. Multiple steady-state simulations can be run at different time steps, but each represents a stable equilibrium. This is the simplest form of modelling to integrate with live data. 

\begin{table}[h]
    \centering
    \begin{tabular}{|l|p{10cm}|}
        \hline
        \textbf{Functionality} & \textbf{Description} \\
        \hline
        Steady State Modelling & Analysis of systems in a stable equilibrium where variables do not change over time. \\
        \hline
        Dynamics & Modelling of time-dependent processes to understand how systems evolve over time. \\
        \hline
        Optimisation/Control & Techniques to find the best operating conditions or control strategies for a system. \\
        \hline
        Hybrid Modelling & Combining different modelling approaches, such as data-driven and first-principles models, to improve accuracy and robustness. \\
        \hline
    \end{tabular}
    \caption{Different pieces of functionality evaluated from the IDAES-PSE framework.}
    \label{tab:functionality}
\end{table}

Dynamic Modelling adds a time dimension to the simulation, allowing changes in the system to propogate through over time. This is useful for predicting how a factory will respond to changes in input conditions, and can be a key decision-making tool for factory operators.

Optimisation and Control techniques can be used to find the best operating conditions for a system, or to develop control strategies that keep the system in a desired state. This enables closed-loop control of a industrial process, where the system can automatically adjust itself to maintain optimal conditions.

Hybrid Modelling implements Machine Learning techniques into the model. IDAES is built on Pyomo, an algebraic modelling language \cite{bynum2021pyomo}, but it is possible to combine machine learning techniques with data-driven modelling libraries such as PySMO. 
This is useful for situations where the system is too complex to model using a single approach, or where data is available but the underlying physics are not well understood for parts of the system. 
The aim of hybrid modelling is to increase generalisation and explainability compared to purely data-driven models, and increase accuracy and adaptability compared to purely first-principles models.

\subsection{Evaluation Criteria}

Each functionality is tested using by developing and evaluating a simple prototype in the IDAES-PSE framework. 

The prototypes are used to evaluate the functionality of IDAES-PSE against three broad categories of use in a factory floor environment: As a standalone tool, when connected with process data, and when connected to control systems. 

\begin{table}[h]
    \centering
    \begin{tabular}{|l|p{10cm}|}
        \hline
        \textbf{Category} & \textbf{Evaluation Criteria} \\
        \hline
        As standalone tools & How well does IDAES-PSE perform the functionality on its own? \\
        \hline
        With process data & Are the tools useful to better process data from a factory? What additions or modifications are needed to make them useful? \\
        \hline
        For control systems & Can the tools be used to improve control systems in a factory? What  is required to do so? \\
        \hline
    \end{tabular}
    \caption{Evaluation criteria for IDAES-PSE functionality.}
    \label{tab:evaluation_criteria}
\end{table}

Evaluating the functionality of IDAES-PSE as a standalone tool gives an indication of the core functionality. This acts as a control, to compare what value is added when connected with live data systems. It acts as an indication of what IDAES functionality is useful for conventional process modelling.

Analysing how the functionality can be used with process data and control systems gives an indication of the potential value of simulation software in a factory environment. 

%Process Data
%Maintenance Data
%Control Systems
% Manual control






\section{Steady State Modelling}

\section{Results \& Discussion}
%Results should be clear and concise.

% TODO: Add a table for each piece of functionality evaluated, showing the results of the evaluation against the criteria.




% Discussion
% This should explore the significance of the results of the work, not repeat them. A combined Results and Discussion section is often appropriate. Avoid extensive citations and discussion of published literature.
%Note: results and Discussion can also be combined in a single section.


\section{Conclusions}
% The main conclusions of the study may be presented in a short Conclusions section, which may stand alone or form a subsection of a Discussion or Results and Discussion section.


\section*{Acknowledgements}
% Collate acknowledgements in a separate section at the end of the article before the references and do not, therefore, include them on the title page, as a footnote to the title or otherwise. List here those individuals who provided help during the research (e.g., providing language help, writing assistance or proof reading the article, etc.).



% References
% Please ensure that every reference cited in the text is also present in the reference list (and vice versa). Unpublished results and personal communications are not recommended in the reference list, but may be mentioned in the text. If these references are included in the reference list they should follow the standard reference style.
\bibliographystyle{apalike}
\bibliography{refs} % Entries are in the refs.bib file



% If there is more than one appendix, they should be identified as A, B, etc. Formulae and equations in appendices should be given separate numbering: Eq. (A.1), Eq. (A.2), etc.; in a subsequent appendix, Eq. (B.1) and so on. Similarly for tables and figures: Table A.1; Fig. A.1, etc.
\begin{appendices}


\end{appendices}


\end{document}