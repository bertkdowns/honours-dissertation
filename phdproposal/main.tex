\documentclass[12pt]{article}
\usepackage[a4paper, portrait, margin=2cm]{geometry}
\usepackage{graphicx} % Required for inserting images
\usepackage[page,toc,titletoc,title]{appendix}
\usepackage{mathptmx}
\usepackage{pdflscape}
\usepackage{pdfpages}
\usepackage{enumitem}
\usepackage{hyperref}
\usepackage{cleveref}
\usepackage{wrapfig}
\usepackage{gensymb}

\title{A Software Engineering Approach to the Design and Application of Digital Twins in Industrial Chemical Processes}

\author{Bert Downs }

\date{October 2024}

\begin{document}

\maketitle

Digital Twin technology is seen as transformative in many industries, as they enable new methods of control, optimisation, scheduling, and design~\cite{walmsley2024adaptive}. The increase in efficiency they promise is particularly attractive in the energy sector, where it can reduce both emissions and cost.

Some key technologies underpining digital twins include:

\begin{itemize}
    \item Traditional Process Simulation techniques, where mathematical physics-based models are used to simulate the behaviour of a system \cite{lee2021idaes}.
    \item Data Collection and Processing techniques, such as SCADA systems or Industrial IoT systems, which enable observability and control \cite{udugama2020role},
    \item Physics-Informed Machine Learning, where machine learning models are combined with traditional physics-based models to improve accuracy and generalisation \cite{karniadakis2021physics}.
\end{itemize}

There is a growing body of research into new machine learning techniques that can be used to accurately model complex systems. New fields of machine learning, such as Operator Networks, are particularly promising as they offer the ability to predict the solution to Ordinary Differential Equations (ODEs) and Partial Differential Equations (PDEs)\cite{lu2019deeponet}, which are commonly used in physics-based models.

These techniques exist as research concepts, but Software Engineering principles have not been applied to implement them in a live industrial environment. This project aims to explore the development of a framework for implementing Digital Twins, with industrial chemical processes providing a case study.

\section*{Research Questions}
% how do you build better software faster?

% These questions need to be more specific, more focused on the software engineering aspects of the project, and more like a hypothesis.
% What do you want to do
% What do you want to prove
% What problem is there, why do you propose your solution

This project aims to answer the following research questions:

\textbf{How does the level of abstraction impact the effort required to design and implement a Digital Twin for an industrial chemical process? To what extent does it enable wider applicability of Digital Twin tools and techniques?}

A Digital Twin platform must implement functionality from many different types of models, in a way that is suitable for real-time systems \cite{cao2021simulation}. Thus each Digital Twin is unique and application-specific. A high level of abstraction is a barrier to customising Digital Twins, but a low level of abstraction reduces reusability. This research question will apply Software Engineering principles to determine where abstraction is beneficial in the design and implementation of a Digital Twin.

%  How can we network multiple machine learning/physics based models together? Can we make these independent but still as accurate as a single model of the entire system? - Model Observability/Interpretability

\textbf{How effective are Operator Networks at surrogate modelling dynamic chemical processes in an online setting?} % for model predictive control?
%% Hypothesis: Operator Networks are a suitable technique for surrogate modelling dynamic systems in a live environment, because they can be updated in real-time, and can handle concept drift.

Operator networks are a new machine learning technology that have the potential to model dynamic systems more accurately than traditional machine learning techniques. This research question aims to adapt existing operator networks to support online learning, and to test their effectiveness in a live industrial environment. 

\textbf{What are the real-world performance benefits associated with implementing Digital Twin Techniques?}

Recent work has outlined a conceptual framework for implementing Digital Twin Technologies \cite{ors2020conceptual}, but there is little research into how they perform in a real-world setting. This research question aims to deploy the Ahuora platform in a live industrial environment, and to measure the performance benefits that these techniques provide. This will provide evidence to industrial partners that these techniques are effective. 

\textbf{What software engineering principles, practices, and frameworks help build more maintainable Digital Twin systems?}
% "Build better software faster" basically means build maintainable software.

The development of the Ahuora platform will provide a case study in building Digital Twin Systems, and the effectiveness of Software Engineering principles and practices. As a suitable architecture is developed, this can be generalised into a software model for building Digital Twins.  

\section*{Proposed Methodology}

% Add dynamic modelling to the Ahuora platform
% Add surrogate modelling to the Ahuora platform
% add optimisation to the Ahuora Platform
% Use operator networks to solve the partial differential equations for optimisation
% Add the required features for control
% 

Literature and technology from the fields of Software Engineering, Data Science, and Chemical Engineering will be reviewed in context of the other fields.
Development will be conducted using Software Engineering principles. Small-scale experiments, case studies, and prototypes will be used to test the effectiveness of different modelling techniques, and their suitability for a live industrial environment. 
This research will extend the Ahuora Platform to support dynamic modelling, surrogate modelling, data-driven modelling, optimisation, and control tooling. 

\section*{Expected Outcome}

This research will enable Digital Twin technologies, Operator Networks, and multi-fidelity modelling to be used in conjunction with one another, outside of their core research communities.

By the end of this project, the Ahuora Digital Twin Platform will support each of these modelling techniques. The Ahuora will be deployed in New Zealand industries, providing real-world performance benefits. This will provide evidence to industrial partners that these techniques are viable, feasible, and desirable, enhancing their uptake by the Industrial Process sector. 

Additionally, this research will provide a case study for how Software Engineering principles can be applied to the development of Digital Twins, and a framework for implementing Digital Twins in other industries.


% \section{Resources}

% Compute, servers to test on.
% Developers
% Data (from a real industrial process)
% Access to the Ahuora platform


% What background research is there?
% All the live data stuff
% all the idaes stuff
% all the surrogate modelling stuff
% all the physics informed stuff
% all the ahuora stuff

% What can we add?
% operator networks, more physics informed, but also a LIVE SYSTEM UPDATES not just historical data

% What is the plan?

% Ahuora makes a good case study as a plaform with a live simulation enviromnent
% but, we can make a more general solution for operator networks, in mathematical modelling, in live data
% we can also look at operator networks, in live data, which goes to mathematical modelling (no neural networks in the mathematical model)
% or maybe operator networks as a live surrogate of a mathematical model that's too slow to run/update in real time

% We can also look at the physics informed stuff, and how that can be used in a operator network. (?maybe)


%https://www.overleaf.com/learn/latex/Bibtex_bibliography_styles
\bibliographystyle{apalike}
\bibliography{refs} % Entries are in the refs.bib file

\end{document}