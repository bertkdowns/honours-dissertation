\documentclass[12pt]{article}
\usepackage[a4paper, portrait, margin=2cm]{geometry}
\usepackage{graphicx} % Required for inserting images
\usepackage[page,toc,titletoc,title]{appendix}
\usepackage{mathptmx}
\usepackage{pdflscape}
\usepackage{pdfpages}
\usepackage{enumitem}
\usepackage{hyperref}
\usepackage{cleveref}
\usepackage{wrapfig}
\usepackage{gensymb}

\title{Evaluating The Effectiveness of IDAES and the Ahuora Simulation Platform for Live Data Processing}

\author{Bert Downs\\
\textit{Ahuora Research Group}\\
\textit{University of Waikato}\\
\textit{Hamilton, New Zealand}\\
\texttt{bd65@students.waikato.ac.nz}}

\date{October 2024}

\begin{document}

\maketitle

\section*{Abstract}
% A concise and factual abstract is required. The abstract should state briefly the purpose of the research, the principal results and major conclusions. An abstract is often presented separately from the article, so it must be able to stand alone. For this reason, References should be avoided, but if essential, then cite the author(s) and year(s). Also, non-standard or uncommon abbreviations should be avoided, but if essential they must be defined at their first mention in the abstract itself.

% Keywords. Immediately after the abstract, provide a maximum of 6 keywords avoiding general and plural terms and multiple concepts (avoid, for example, "and", "of"). Be sparing with abbreviations: only abbreviations firmly established in the field may be eligible. These keywords will be used for indexing purposes.

\cite{CITATION_NEEDED}

\section{Introduction}
% State the objectives of the work and provide an adequate background, avoiding a detailed literature survey or a summary of the results.

Traditional factory control systems are based on feedback loops that use sensor data to adjust the factory's state. However, these systems are limited in their ability to predict future states and optimise factory performance. Digital Twinning is a new approach that combines live factory data, historical state, mathematical modelling, and data-driven modelling to create a digital replica of the factory. 
However, in existing literature, most examples of digital twinning focus on specific situations, rather than providing a general framework for creating digital twins. Many remain at the proof-of-concept stage, and have not yet been deployed in industrial settings.



%Experimental procedure
%Provide sufficient details to allow the work to be reproduced by an independent researcher. Methods that are already published should be summarized, and indicated by a reference. If quoting directly from a previously published method, use quotation marks and also cite the source. Any modifications to existing methods should also be described.
%This section can contain Material and methods and Theory/calculation. A Theory section should extend, not repeat, the background to the article already dealt with in the Introduction and lay the foundation for further work. In contrast, a Calculation section represents a practical development from a theoretical basis.

\section{Results \& Discussion}
%Results should be clear and concise.

% Discussion
% This should explore the significance of the results of the work, not repeat them. A combined Results and Discussion section is often appropriate. Avoid extensive citations and discussion of published literature.
%Note: results and Discussion can also be combined in a single section.


\section{Conclusions}
% The main conclusions of the study may be presented in a short Conclusions section, which may stand alone or form a subsection of a Discussion or Results and Discussion section.


\section*{Acknowledgements}
% Collate acknowledgements in a separate section at the end of the article before the references and do not, therefore, include them on the title page, as a footnote to the title or otherwise. List here those individuals who provided help during the research (e.g., providing language help, writing assistance or proof reading the article, etc.).



% References
% Please ensure that every reference cited in the text is also present in the reference list (and vice versa). Unpublished results and personal communications are not recommended in the reference list, but may be mentioned in the text. If these references are included in the reference list they should follow the standard reference style.
\bibliographystyle{apalike}
\bibliography{refs} % Entries are in the refs.bib file



% If there is more than one appendix, they should be identified as A, B, etc. Formulae and equations in appendices should be given separate numbering: Eq. (A.1), Eq. (A.2), etc.; in a subsequent appendix, Eq. (B.1) and so on. Similarly for tables and figures: Table A.1; Fig. A.1, etc.
\begin{appendices}


\end{appendices}


\end{document}