\chapter{Prototype: Live Data Simulation} \label{sec:simulationprototype}

\section{Purpose}

The purpose of the prototype was to demonstrate how the Ahuora Simulation Platform could be integrated with a real-time data processing system. It was designed to be a minimum-viable product, demonstrating the feasibility of the architecture developed in the research stage. As appropriate, it could form the basis of a more comprehensive framework, or provide insights into the requirements of a future system.

\section{Method}

\begin{wrapfigure}{r}{0.5\textwidth}
    \centering
    \includegraphics[width=0.5\textwidth]{swaggerprops.png}
    \caption{Swagger definition of API endpoints for updating properties.}
    \label{fig:swaggerendpoints}
\end{wrapfigure}

The Ahuora Simulation Platform has a REST API that is used internally to communicate between the frontend and the backend. This API already includes endpoints for updating properties, and retrieving properties after a simulation. This api could also be used for real-time data processing systems to communicate programmatically with the platform.

Each property field has a unique ID, generated by the database. The flowsheet needs to be set up ahead of time with the relevant unit operations, and properties. To solve the flowsheet repeatedly based on real-time data, the properties in the flowsheet can be updated to reflect the real-time state programmatically, and then a call can be made to the API endpoint to solve the flowsheet. Then, relevant properties can be queried to get the results of the simulation.

\begin{figure}
    \centering
    \includegraphics[width=0.8\textwidth]{property_ui.png}
    \caption{Prototype UI designs for setting a property as a ``real-time'' property.}
    \label{fig:property-ui}
\end{figure}

As the property IDs are generated by the database, it does not make sense to hard-code them into the real-time data processing system. There should be a more generic way of referring to properties that need to be updated. Some prototyping of a UI method to set properties as ``real-time'' in the existing interface was done, as shown in \cref{fig:property-ui}. This would allow the user to select which properties should be updated in real-time, and provide a way to refer to them in the real-time data processing system. However, ultimately we decided not to implement this yet, partially because it did not provide a clear separation between the engineers workflow and the operators workflow, and partially because it was not necessary for the prototype.

Instead, we used a simple definitions file to map the properties in the Ahuora Simulation Platform to the properties in the real-time data processing system. As shown in \cref{fig:live-constants}, the file included the unit operation name, the property key (used by the Ahuora Simulation Platform to determine property types), and the sensor ID from the real-time data processing system. The data processing platform could use this information to find the property IDs in the Ahuora Simulation Platform, and update them with the real-time data.

\begin{figure}
    \centering
    \includegraphics[width=0.8\textwidth]{live_constants.png}
    \caption{Definitions file for mapping properties in the Ahuora Simulation Platform to the real-time data processing system.}
    \label{fig:live-constants}
\end{figure}

Rather than truly connecting it to real-time data, for now the prototype used a CSV file with dummy data in the same format as the real-time data collected in \cref{sec:heatpumpcollection}. This was simply to make debugging and testing easier; the process would have been the same if the data was being read from the sensors live.

This prototype also used a very simple model: a pump with an inlet stream and an outlet stream. The power used by the pump was calculated based on the  ``live'' data, and the inlet streams were already specified. The simulation would be used to calculate the outlet pressure and temperature of the pump. This was a simple model, but it was sufficient to demonstrate the feasibility of the architecture.

% TODO: Add image: Running the simulation on the CSV file.

When the script was run, for each new data point, the script would update the properties in the Ahuora Simulation Platform, and then call the API to solve the simulation. The results of the simulation were then printed to the console. 

This worked suprisingly well, and was able to be done in only around 300 lines of code. It was implemented in Rust, using an API client SDK generated from the OpenAPI specification of the Ahuora Simulation Platform. This made it fully type-safe and reliable. Because of the configuration file format, it would be trivial to add a longer list of sensors and calculated properties, as would be required in more complex properties.

\section{Insights} \label{sec:prototypeinsights}

This prototype significantly affected the focus of future work. The prototype demonstrated that the Ahuora Simulation Platform could be integrated with a real-time data processing system with relatively little effort, as long as there was a standardised API. Sure, depending on the type and quality of the sensor data, there may need to be some data cleaning and preprocessing, but that is use-case specific business logic. Hence, there may be little need to develop a standardised data preprocessing pipeline or service for the Ahuora Digital Twin Platform.
To confirm this, the heat pump dryer will be used as a case study in linking a more complex system (with custom business logic required) to the Ahuora Simulation Platform.

This approach, which closely follows the architecture described in \cref{fig:architecture} in \cref{sec:researchconclusions}, also outsources the processing, visualisation, and control actions to third-party systems, as all data would be stored in the factory's existing knowledge base after processing, completely external to all Ahuora Systems. In some ways, this is a good approach: it could be easy to make a "headless" version of the Ahuora Simulation platform that can be deployed with a single frozen model, into a factory's existing systems. This would be good for stability and reliability reasons, limiting the complexity of the system. This is the unix ``do one thing and do it well" philosophy applied to software architecture.

However, there are disadvantages to this approach. One of the key value propositions of the Ahuora Digital Twin Platform is that it provides one place to define a factory's architecture, and then multiple types of analysis can be used on it. The same structures could be helpful for automatically creating visualisations of the real-time state of the factory, fault diagnostics, control, and surrogate modelling. As the platform currently stands, it cannot be considered a ``Digital Twin'', true digital twins include multiple fidelities of simulation, and dynamic ``state'' that adapts to real-world conditions via a feedback loop. The next steps in development will need to balance these two approaches, to find an architecture that includes the advantages of each.
