\chapter{Methodology}
% helicopter view, structure of the report
%TODO: Make this a bit more detailed, going over each of the sections

Software Engineering principles are notoriously hard to follow in academia \cite{connolly2023software}, partially because of of competing criteria for success. 
In academia, the criteria is publication of results, but in Software Engineering, creating and publishing a quality piece of software is quintessential to success. 
Additionally, since this project is conducted as part of a wider effort to improve the Ahuora Platform, research and development often had to account for a changing environment. 
Thus Agile principles were followed for both research and development, focusing on implementing one feature at a time. 

This project largely follows the ``Action Research'' methodology reviewed by Wohlin et al.~\cite{wohlin2021guiding}. The Action Research methodology is an iterative process, which includes five phases:

\begin{enumerate}
    \item Diagnosing - Identifying a problem to be addressed. The larger problem has already been identified in the Introduction and Literature Review. Each chapter also starts by identifying specific research questions, or the purpose of implementing a feature.
    \item Action Planning - Deciding what approach will best solve the problem. This is done by deciding on a prototype to build or a feature to implement, which is done at least once in each chapter.
    \item Action Taking - This involves setting the planned actions into practice - by building a prototype for research purposes, or implementing a feature in the Ahuora Platform.
    \item Evaluating - Studying and discussing the consequences of an action. The effectiveness of a prototype or feature, along with any other insights from the development process, is discussed at the conclusion of each chapter.
    \item Specifying/learning - Identifying general findings related to the problem under study. This is done explicitly in some chapters, such as by identifying the requirements of different types of users. The report is concluded with a chapter specifically devoted to generalising the findings of each previous chapter into a theoretical framework that can be used for implementing Digital Twins, and an implementation plan for the Ahuora Platform. 
\end{enumerate}


\section{Overview of Work}

\begin{table}[ht]
    \centering
    \caption{Overview of chapters}
    \label{tab:research_chapters}
    \begin{tabular}{|L{4.5cm}|p{11.5cm}|}
    \hline
    \textbf{Chapter} & \textbf{Purpose} \\
    \hline
    Data Collection \mbox{(\Cref{sec:heatpumpcollection})}& Understand the process of collecting data from a thermodynamic system, using a Heat Pump Dryer as a test case. \\ \hline
    Simulation Technologies \mbox{(\Cref{sec:architectureresearch})}& Identify potential analysis techniques for a Digital Twin system, and what is required to make use of them. \\\hline
    Prototyping \mbox{(\Cref{sec:simulationprototype})}& Experiment with linking live data to a simple simulation in the Ahuora Platform. \\\hline
    Recording History \mbox{(\Cref{sec:history})}& Adding the ability to store past solves in the Ahuora Digital Twin Platform. \\\hline
    Data Preprocessing \mbox{(\Cref{sec:datapreprocessing})}& Functionality for parameterising solves, to make it easier to develop Digital Twins. \\
    \hline
    \end{tabular}
\end{table}

Each chapter has a distinct objective, yet each piece of work contributes to the same high-level objectives. \Cref{tab:research_chapters} summarises the specific focus of each chapter. 
Chapters \ref{sec:heatpumpcollection} \& \ref{sec:architectureresearch} focus on understanding the core tools and platforms that relate to Digital Twin solutions, establishing a holistic long-term view of the problem. These chapters are more research focused.

Chapter \ref{sec:simulationprototype} assesses the conclusions from the previous chapters, by building a prototype live data collection system. A theoretical framework is presented that can be used generally to inform Digital Twin development efforts. 

Chapter \ref{sec:history} develop some features in the Ahuora Simulation Platform that implement characteristics discussed in the earlier chapters. The features discussed in this section are now integrated into the Ahuora Platform codebase.

A case study of a heat pump dryer was used throughout the report to evaluate the prototypes built and features developed.  This was achieved by developing a model of a heat pump dryer in the Ahuora Simulation Platform, and integrating real-time sensor data into the model. The case study was used to evaluate the feasibility of the framework, and to identify areas for future work.
