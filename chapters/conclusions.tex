\chapter{Conclusions}

% This chapter summarises your project, including a concise and significant summary of the R&D project findings, and contributions. 

% Present the outcome, and the arguments.
% Link back to research questions

%\section{Objectives \& Outcomes}


%\section{Research}

The research conducted in this project has identified where the key areas of work are in developing a Digital Twin Platform for Ahuora. 
This has resulted in a framework being proposed for the system, wherein the Digital Twin Platform is built on top of a data processing pipeline and a simulation platform. 
It has been tested via prototyping to show that it is appropriate for those who are involved in both design and operation of the factory, from a data or chemical engineering perspective.

%\section{Development}

This has been implemented in the Ahuora Digital Twin Platform for steady-state modelling. 
The ability to store solve history was added to visualise the results of simulations, and functionality to preprocess data was added to make it possible to convert input data to the format required in chemical modelling. 

As a test case, a model of a heat pump dryer was created and the platform was used to simulate the dryer's performance. 
This showed that the platform is capable of simulating a factory's performance, and identified some edge cases where care must be taken to specify the model in a way that it can be reliabily solved.

This work addressed some of the main challenges in implementing digital twins in the industry, namely, the complexity and cost of building a Digital Twin system from scratch. Through the continued development of Digital Twin Platforms as specified in this report, Project Ahuora's goal to decarbonise the process heat sector will be fully realised.

%\section{Future Work}
% Strengths, limitations, impacts, what could have been better, etc


\section{Future work}
% TODO: Update this section from the interlude
Future work could focus on adding support for dynamic models, hybrid modelling, and optimisation to the Ahuora Platform. Then the live data processing system could be developed to support these features. 
This would increase the usefulness of the platform for factory operation and control. 


The research conducted so far has provided the context required to build a high-level roadmap of future development. This is provided in Appendix \ref{app:roadmap}.



Further work should also be performed on the feasibility of creating a standalone `Deployment' version of the Ahuora Platform, specifically focused on control and optimisation. The research presented in this report implies that such a product would better fit the constraints of a real factory, in cases where cloud platforms may not be appropriate.

%\section{Impact Statement}


% evaluates the broader impact of the project outcome, either quantitatively or qualitatively (e.g., social, economic, environmental, health, safety, legal, ethical, and/or cultural issues)
%\section{Impact}


