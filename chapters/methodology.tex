\chapter{Methodology}
% helicopter view, structure of the report

This project was conducted as part of a real-world, collaborative effort to build the Ahuora Digital Twin platform, so the methodology followed engineering principles. The focus of the project moved between improving the Simulation Platform to support more modelling techniques, and developing a live data processing system that could be integrated into the platform.

Agile methods were followed, with a focus on iterative development and continuous feedback. Each chapter of the report represents a full ``sprint'': identifying a problem to solve, developing a solution, and evaluating the solution. Each chapter has a distinct objective, which varies based on the state of the project. Some are exploratory, focusing on research, understanding needs, and long-term plans. Other objectives are more concrete, focusing on developing and using the system. 


% IDK if any of this is relevant or not

\section{Research}

The research stage involved small-scale experiments in the IDAES framework to test the feasibility of integrating live data processing techniques into the Ahuora Digital Twin Platform. This followed directly from the Literature Review, which identified a variety of modelling techniques that could be used in a Digital Twin Platform. The research stage was used to identify the requirements for the Ahuora Digital Twin Platform, and to develop a high-level architecture for the platform that would support these requirements. This focused on the long-term requirements of the platform, rather than the current capabilities of the Ahuora Simulation Platform.

\section{Development}

The development stage involved developing a pilot implementation of the framework. This was done in conjunction with the Ahuora Simulation Platform, following the architecture developed in the research stage. However, it did not include all the features discussed in the research section, due to the limitations of the current platform. Only the features supported by the current version of the Ahuora Simulation Platform were implemented.

\section{Evaluation}

Evaluation was conducted throughout the research and development process, by testing the prototypes and implementation in an example scenario, forming a simple case study. This was done by developing a model of a heat pump dryer in the Ahuora Simulation Platform, and integrating real-time sensor data into the model. The case study was used to evaluate the feasibility of the framework, and to identify areas for future work.

